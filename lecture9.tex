\section{Devátá přednáška}
\subsection{Geometrie}

\begin{example}[Příčka mimoběžek]
    Mějme přímky $p$ a $q$ a bod $K$:
    \begin{align*}
        p&:\;\;\;A=[2, 1, 1],\,B = [3, 2, 6]\\
        q&:\;\;\;C=[2, -5, -1],\,D = [0, -4, 3]\\
        K &= [-3, 0, 3]
    \end{align*}
    Bodem $K$ veďte příčku $r$ k mimoběžkám $p,\,q$.

    Pro oveření že $p$ a $q$ jsou mimoběžky vytvoříme matici $P$ a určíme její hodnost.
    \begin{align*}
        P &= \begin{pmatrix}
            \vec{AB}\\
            \vec{CD}\\
            \vec{AC}
        \end{pmatrix} =
        \begin{pmatrix}
            1 & 1 & 5\\
            -2 & 1 & 4\\
            0 & -6 & -2
        \end{pmatrix} \eqop{}
        \begin{pmatrix}
            1 & 1 & 5\\
            0 & 3 & 14\\
            0 & -6 & -2
        \end{pmatrix} \eqop{}
        \begin{pmatrix}
            1 & 1 & 5\\
            0 & 3 & 14\\
            0 & 0 & 26
        \end{pmatrix} = R\\
        h(R) &= 3 = h(P)
    \end{align*}
    Matice má plnou hodnost $3$ a určitě se tedy jedná o mimoběžky.

    Máme tedy dvě přímky, které jsou mimoběžné a bod K, nyní hledáme
    přímku, která prochází bodem $K$ a zároveň má nějaké body $P$, $Q$,
    které tvoří průsečíky s přímkami $p$, $q$.

   Pomocí vektoru $\vec{AB}$ a bodu $A$ vyjádříme bod $P$ pomocí parametru $t$:
   $$P = [2 + t, 1 + t, 1 + 5t]$$
   Podobně pomocí vektoru $\vec{CB}$ a bodu C vyjádříme bod $Q$ pomocí parametru $s$:
   $$Q = [2 - 2s, -5 +s, -1 + 4s]$$

   Nyní vyjádříme vektory $\vec{KP}$ a $\vec{KQ}$ o kterých víme, že mají být lineárně závislé.
   \begin{align*}
       \vec{KP} &= [5 + t, 1 + t, -2 + 5t]\\
       \vec{KQ} &= [5 - 2s, -5 + s, -4 + 4s]
   \end{align*}
   A z toho, že vektory $\vec{KP}$ a $\vec{KL}$ mají být lineárně závislé můžeme odvodit:
   \begin{align*}
       \vec{KP} &= l \cdot \vec{KQ}\\
       (5 + t, 1 + t, -2 + 5t) &= l \cdot (5 -2s, -5 + s, -4 + 4s)
   \end{align*}
    Uvážíme li proměnné $t, l, ls$, potom dostaneme soustavu 3 lineárních rovnic o
    třech neznámých.
    \begin{align*}
    \begin{pmatrix}[ccc|c]
        1 & -5 & 2 & -5\\
        1 & 5 & -1 & -1\\
        5 & 4 & -4 & 2
    \end{pmatrix}
    \end{align*}

    Z této soustavy nám k vyřešení úlohy stačí vyjádřit neznámou $k$, nebo $s$,
    k čemuž se nám hodí cramerovo pravidlo:
    \begin{align*}
        t &= \frac{
            \begin{vmatrix}
                -5 & -5 & 2\\
                -1 & 5 & -1\\
                2 & 4 & -4\\
            \end{vmatrix}
        }{
            \begin{vmatrix}
                1 & -5 & 2\\
                1 & 5 & -1\\
                5 & 4 & -4\\
            \end{vmatrix}
        } = \frac{100 - 8 + 10 - (20 + 20 -20)}{-20 + 8 + 25 - (50 -4 + 20)} =
        \frac{78}{-53}
    \end{align*}
    A můžeme vyjádřit vektoř $\vec{KT}$:
    $$KP = (\frac{265 - 82}{53}, \frac{53 - 82}{53}, \frac{-106 - 410}{53}) =
    (\frac{183}{53}, \frac{-29}{53}, \frac{-516}{53})$$
    Nás zajímá pouze směr tohoto vektoru, můžeme si pomocí něj tedy určit směrový vektor,
    který bude vhodně vynásobený:
    $$\vec{u} = (183, -29, -516)$$
\end{example}

\subsubsection{Úlohy o vzdálenostech}
Vzdálenost dvou bodů můžeme vyjádřit jako
$$d(A, B) = ||\varphi (A,B)|| = \sqrt{\pi(\varphi(A, B)),\, \varphi(A, B)} =
\sqrt{(b_1 - a_1)^2 + \ldots + (b_n - a_n)^2}$$

\subsubsection{Vzdálenost bodu a přímky}
Máme nějaký bod $A$ a přímku $p$ a chceme zjistit jejich vzájemnou
vzdálenost $d(A, p)$.
Pro tento problém existuje obecný vzorec. Nám jde ale spíše
o to, jak takový vzorec odvodit, než ho jen použít.

Možnost řešení: vezmeme rovinu $\rho$ která bude kolmá k $p$ a zároveň bude procházet
bodem $A$, najdeme průsečík $K$ roviny $\rho$ a přímky $p$ (který bude jediný) a spočítáme
vzdálenost bodů $K$ a $A$.

\subsubsection{Vzdálenost bodu a roviny}
V případě vzdálenosti bodu a roviny můžeme použít obdobný přístup.
Máme nějaký bod $A$ a rovinu $\rho$, chceme spočítat $d(A, \rho)$.

K rovině najdeme kolmou přímku $p$,
která prochází bodem $A$. Najdeme průsečík $K$ přímky $p$ a roviny $\rho$,
a spočítáme vzdálenost bodů $K$ a $A$.

\subsubsection{Vzdálenost dvou rovnoběžných přímek}
Máme dvě rovnoběžky $p$ a $q$ a chceme vypočítat jejich vzájemnou
vzdálenost $d(p, q)$.

Na jedné z rovnoběžek si můžeme zvolit libovolný bod a postupovat
stejně jako při výpočtu vzdálenosti bodu a přímky.

\subsubsection{Vzdálenost prímky od roviny}
Máme přímku $p$ a rovnoběžnou rovinu $\rho$ a chceme spočítat jejich vzájemnou
vzdálenost $d(p, \rho)$.

Na přímce $p$ můžeme zvolíme libovolný bod a postupovat stejně
jako při výpočtu vzdálenosti bodu a roviny.

\subsubsection{Vzdálenost dvou rovnoběžných rovin}
Máme dvě rovnoběžné roviny $\rho$ a $\sigma$ a chceme spočítat
$d(\rho, \sigma)$.

Na jedné z rovin si můžeme zvolit libovolný bod a postupovat
stejně jako při výpočtu vzdálenosti roviny a bodu.

\subsubsection{Vzdálenost dvou mimoběžných přímek}
Máme dvě mimoběžné přímky $p$ a $q$ a chceme spočítat jejich
vzájemnou vzdálenost $d(p, q)$.

Vezmeme dvě navzájem rovnoběžné roviny a v nichž leží přímky $p$ a $q$.
Normálový vektor rovin můžeme najít jako cross product směrových vektorů
přímek $p$ a $q$. A následně můžeme postupovat jako v případě vzdálenosti dvou
rovin.

\subsection{Úlohy o úhlech}
Nejprve si připomeňme co je to odchylka vektorů.
Odchylka vektorů $\vec{u}$ a $\vec{v}$ se definuje jako
$\varphi \in \langle 0, \pi \rangle$, které splňuje:
$$cos\,\varphi = \frac{\vec{u} \cdot
\vec{v}}{||\vec{u}|| \cdot ||\vec{v}||}$$

\subsubsection{Vzájemný úhel dvou přímek}
Úhel dvou různoběžných přímek budeme brát jako
menší z úhlů, které mezi sebou svírají, a můžeme ho určit jako:
$$cos\,\varphi = \frac{|\vec{u} \cdot \vec{v}|}{||\vec{u}|| \cdot ||\vec{v}||}$$
kde $||$ v čitateli vyjadřuje absolutní hodnotu.

V případě že přímky budou mimoběžné, tak jejich úhel vypočítáme
úplně stejně a vždycky to bude pravý úhel.

\subsubsection{Vzájemný uhel přímky a roviny}
Máme rovinu $\rho$ a přímu $p$, které tuto rovinu protíná.
A chceme spočítat jejich vzájemný úhel.

Zkonstruujeme pomocnou rovinu $\sigma$, která je kolmá na rovinu
$\varphi$ ve které leží přímka $p$. K vytvoření této roviny stačí vzít bod
na přímce $p$, směrový vektor přímky $p$ a normálový vektor roviny $\rho$.
Roviny $\rho$ a $\sigma$ se poté protnou v přímce $q$
A nyní stačí spočítat vzájemný úhel přímek $p$ a $q$.

\subsubsection{Vzájemný úhel dvou rovin}
Úhle dvou roviny můžeme vypočítat jako odchylku
normálových vektorů, které vezmeme tak, aby jejich odchylka
byla $\leq \frac{\pi}{2}$

\subsubsection{Souhrnný příklad}
\begin{example}[Souhrnný příklad]
    Rovina $\rho$ prochází body $A = [2, 1, 2]$, $B = [4, 0, 1]$ a $C = [-3, 3, 4]$.
    Rovina $\sigma$ je rovnoběžná s $\rho$ a má od ní vzdálenost $4\sqrt{2}$.
    Na rovině $\sigma$ leží bod $S = [5, 0, ?]$

    Spočtěte vzdálenost $B$ a $S$.

    Nejprve si vyjádříme obecnou rovnici roviny $\rho$:
    \begin{align*}
        x &= 2 + 2t -5s \\
        y &= 1 - 1t + 2s\\
        z &= 2 - 1t + 2s\\
        \\
        y - z +1 &= 0
    \end{align*}
    Normálový vektor $\vec{n}$ k rovině $\sigma$ je tedy $\vec{n} = (0, 1, -1)$ a
    rovinu $\rho$ pomocí něj můžeme vyjádřit jako:
    $$y - z + d = 0$$
    Musíme však ještě vyjádřit neznámou $d$.

    Normálový vektor $\vec{n}$ upravíme tak, aby jeho délka byla rovna vzdálenosti obou rovin.
    \begin{align*}
        ||\vec{n}|| &= \sqrt{0^2 + 1^2 + (-1)^2} = \sqrt{2}\\
        \vec{n_1} &= 4 \cdot \vec{n} = (0, 4, -4)\\
        \vec{n_2} &= -4 \cdot \vec{n} = (0, -4, 4)\\
    \end{align*}
    Nyní přičteme $\vec{n_1}$ k nějakému bodu $E = [0, 0, 1]$, který leží na $\rho$ a dostaneme bod,
    který leží na $\sigma$:
    \begin{align*}
        S_1 &= R + \vec{n_1} = [0, 4, -3]\\
        S_2 &= R + \vec{n_2} = [0, -4, 5]
    \end{align*}
    A dosazením do rovnice pro rovinu $\sigma$ můžeme získat hodnotu neznámé $d$, pro
    který máme dvě možná řešení.
    \begin{align*}
       \sigma_1&:\; y - z - 7 = 0\\
       \sigma_2&:\; y - z + 9 = 0
    \end{align*}
    Nyní můžeme podle obecné rovnice pro rovinu $\sigma$ vyjádřit bod $S$.
    \begin{align*}
        S_1 &= [5, 0, -7]\\
        S_2 &= [5, 0, 9]
    \end{align*}

    A teď už jen spočítáme vzdálenosti bodů jak je požadování v zadání.
    \begin{align*}
        d(B, S_1) &= \sqrt{1 + 0 + 64} = \sqrt{65}\\
        d(B, S_2) &= \sqrt{1 + 0 + 64} = \sqrt{65}
    \end{align*}
\end{example}

\subsection{Formy}
Uvažujme dva vektorové prostory $\mathcal{V}$ a $\mathcal{W}$, obecně nad nějakým
polem $F$. A uvažujme množinu všech homomorfismů z $\mathcal{V}$ do $\mathcal{W}$, kterou
budeme značit jako:
$$Hom(\mathcal{V}, \mathcal{W})$$
Tyto homomorfismy můžeme sčítat, mějme nějaké dva konkrétní homomorfismy
$\alpha, \beta \in Hom(\mathcal{V}, \mathcal{W})$,
potom můžeme zavést operaci sčítání jako:
$$(\alpha + \beta)(\vec{v}) = \alpha(\vec{v}) + \beta(\vec{v})$$
Kde znak $+$ na levé straně představuje novou operaci sčítání homomorfismů,
kterou právě definujeme a $+$ na pravé straně představuje sčítání ve vektorovém prostoru
$\mathcal{W}$.

Takto definované sčítání homomorfismů splňuje axiomy Abelovské grupy
(asociativita, neutrální prvek, inverzní prvek, komutativita).
Asociativita a komutativita zjevně platí a vyplývá z asociativity a komutativity
ve vektorovém prostoru $\mathcal{W}$.

O něco zajímavější je neutrální
prvek, musíme najít takový homomorfismus, který bude tvořit neutrální prvek.
A bude jím nulový homomorfismus o (homomorfismus, který libovolný vektor zobrazí
na nulový vektor).

Inverzním prvkem je potom opačný homomorfismus.

Podobně můžeme definovat operaci pro násobení homomorfismu skalárem:
$$(k \cdot \alpha)(\vec{v}) = k \cdot \alpha(\vec{v})$$

A takto definované násobení skalárem zjevně splňuje podmínky
pro násobení skalárem z definice vektorového prostoru.

$Hom(\mathcal{V}, \mathcal{W})$ v kombinaci s výše definovanými operacemi $+$ a $\cdot$
tedy tvoří vektorový prostor nad nějakým tělesem $F$.

\begin{example}[Homomorfismus]
    \begin{align*}
        \mathcal{V} &= \mathbb{R}^2\\
        \mathcal{W} &= \mathbb{R}^3\\
        \alpha(\vec{v}) &= \alpha((v_1, v_2)) =
        (\alpha_{11} \cdot v_1 + \alpha_{12} \cdot v_2,\,
        \alpha_{21} \cdot v_1 + \alpha_{22} \cdot v_2,\,
        \alpha_{31} \cdot v_1 + \alpha_{32} \cdot v_2)
    \end{align*}
    V tomto případě tedy bude celkem 6 souřadnic homomorfismu a jedná se tedy
    o vektorový prostor s dimenzí 6. Obecně pak platí,
    že dimenze homomorfismu je rovna součinu dimenzí obou vektorových prostorů.
    $$dim\,\alpha = dim\,\mathcal{V} \cdot dim \mathcal{W} = 2 \cdot 3 = 6$$
\end{example}

Uvažujme specialní případ, kdy $\mathcal{W} = F$.
Máme tedy $Hom(\mathcal{V}, F)$, což je opět vektorový prostor, a protože
toto je dosti specialní, že cílový prostor je přímo to pole, tak má i specialní
označení $\mathcal{V}^*$ a říkáme mu dualní prostor
$$\mathcal{V}^* = Hom(\mathcal{V}, F)$$
Jeho prvky jsou samozřejmě vektory, ale budeme je nazývat lineární formy.

Dále budeme pro běžné vektory používat značení s šipkou jako doposud, například:
$$\vec{u}, \vec{v},\ldots \in \mathcal{V}$$
Ale pro vektory, které jsou linearními formami budeme používat řecká písmena:
$$\alpha, \beta, \ldots \in \mathcal{V}^*$$

Formu můžeme aplikovat na nějaký vektor:
$$\alpha(\vec{u})$$
Výsledkem této operace pak bude nějaké číslo, respektive prvek nějakého pole F,
nad kterým pracujeme.

Dimenze dualního prostoru je stejná jako dimence původního protostu, protože
dimenzi lze opět vyjádřit jako součin dimenzí jednotlivých prostorů a dimenze $F$
je 1. Dimenzi $\mathcal{V}$ tedy pouze vynásobíme jedničkou a dostaneme opět stejnou dimenzi.
$$dim\,\mathcal{V}^* = dim\, \mathcal{V} \cdot dim\,F = dim\,\mathcal{V} \cdot 1 = dim\,\mathcal{V}$$

Nyní si můžeme klást zajímavou otázku: Když budeme mít nějaký vektorový prostor $\mathcal{V}$
s nějakou bází, tak jakou bázi má k němu dualní prostor $\mathcal{V}^*$?

Uvažujme vektorový prostor $\mathcal{V}$ s bází $\mathcal{B}$
\begin{align*}
    \mathcal{V}\;\mathcal{B} = \{\vec{v_1}, \ldots, \vec{v_n}\}\\
\end{align*}
Zřejmě $dim \mathcal{V} = |\mathcal{B}| = n$
Potom bázi duálního prostoru $\mathcal{V}^*$ označme $\alpha_1, \ldots, \alpha_n$
a můžeme ji vzít takto:
\begin{align*}
    \alpha_i(\vec{v_j}) = \delta_{ij}
\end{align*}
Kde $\delta$ představuje Kroneckerovo delta, tedy 1 pokud $i=j$ a 0 v případě že $i \neq j$.
Což zjevně není jednoznačné a můžeme mít více možností.

Pak můžeme libovolnou linearní formu $\varphi \in \mathcal{V}^*$ vyjádřit jako linearní
kombinaci prvků báze:
$$c_1 \cdot \alpha_1 + \ldots + c_n \cdot \alpha_n$$
Stačí vzít $c_i$ jako $\varphi(\vec{v_i})$.
\begin{align*}
    \varphi(\vec{v_1}) &= c_1 \alpha_1(\vec{v_1}) + \ldots + c_n \alpha_n(\vec{v_n}) =
    \varphi(\vec{v_1}) \delta_{11} + \varphi(\vec{v_2}) \delta_{12} + \ldots + \varphi(\vec{v_m}) \delta_{1n}
    = \varphi(\vec{v_1}) + 0 +\ldots + 0 = \varphi(\vec{v_1})\\
    \varphi(\vec{v_2}) &= \varphi(\vec{v_2}) \\
    \varphi(\vec{v_3}) &= \varphi(\vec{v_3})\\
    \vdots
\end{align*}
Kroneckerovo delta nám tedy zajistí, že formu $\varphi$ skutečně vyjádříme jako linearní kombinaci
bázových forem.

\begin{example}
    Vektorový prostor $\mathcal{V} \subseteq \mathbb{R}^5$ je dán vztahy:
    \begin{align*}
        v_1 &= v_2\\
        v_1 + v_2 + v_3 &= v_4 + v_5
    \end{align*}
    Určete bázi a dimenzi $\mathcal{V}$ a najděte bázi $\mathcal{V}^*$.

    $$\mathcal{V} = \{(a, a, b, c, 2a + b -c) \forall a, b, c \in \mathbb{R}\}$$
    Dimenze vektorového prostoru $\mathcal{V}$ je  tedy 3.
    A jeho bázi můžeme vyjádřit jako:
    $$\mathcal{B} = \big( (1, 1, 0, 0, 2), (0, 0, 1, 0, 1), (0, 0, 0, 1, -1) \big)$$

    Získali jsme tedy bázi prostoru $\mathcal{V}$ a nyní z ní chceme sestrojit bázi duálního
    prostoru $\mathcal{V}^*$.

    Báze dualního prostoru se v tomto případě bude skládat ze tří forem $\alpha_1,\,\alpha_2,\,\alpha_3$.

    Nejprve vyjádříme $\alpha_1$.
    \begin{align*}
        \alpha_1(\vec{v_1}) &= 1\\
        \alpha_1(\vec{v_2}) &= 0\\
        \alpha_1(\vec{v_3}) &= 0
    \end{align*}
    Z toho můžeme odvodit následujicí rovnice:
    \begin{align*}
        \alpha_{11} \cdot 1 + \alpha_{12} \cdot 1 + \alpha_{13} \cdot 0 + \alpha_{14} \cdot 0 + \alpha_{15} \cdot 2 &= 1\\
        \alpha_{11} \cdot 0 + \alpha_{12} \cdot 0 + \alpha_{13} \cdot 1 + \alpha_{14} \cdot 0 + \alpha_{15} \cdot 1 &= 0\\
        \alpha_{11} \cdot 0 + \alpha_{12} \cdot 0 + \alpha_{13} \cdot 0 + \alpha_{14} \cdot 1 + \alpha_{15} \cdot (-1) &= 0
    \end{align*}
    Nyní stačí najít libovolné řešení této soustavy a tím vyjádřit $\alpha_1$.
    Parametricky můžeme řešení popsat jako:
    $$(1-q-2p,\, q,\, -p,\, p,\, p)$$
    Můžeme $p$ a $q$ zvolit jako 0, potom dostaneme jedno z možných řešení:
    $$(1, 0, 0, 0, 0, 0)$$
    Potom můžeme říct, že první forma funguje takto:
    $$\alpha_1(\vec{u}) = u_1$$

    Velmi podobně můžeme postupovat u ostatních dvou forem.
    Pro $\alpha_2$:
    \begin{align*}
        \alpha_{21} \cdot 1 + \alpha_{22} \cdot 1 + \alpha_{25} \cdot 2 &= 0\\
        \alpha_{23} \cdot 1 + \alpha_{25} \cdot 1 &= 1\\
        \alpha_{24} \cdot 1 + \alpha_{25} \cdot (-1) &= 0
    \end{align*}
    A zase stačí najít alespoň jedno řešení této soustavy, například:
    $$(0, 0, 1, 0, 0, 0)$$
    Druhá bázová forma tedy může být:
    $$\alpha_2(\vec{u}) = u_3$$

    Pro $\alpha_3$:
    \begin{align*}
        \alpha_{31} \cdot 1 + \alpha_{32} \cdot 1 + \alpha_{35} \cdot 2 &= 0\\
        \alpha_{33} \cdot 1 + \alpha_{35} \cdot 1 &= 0\\
        \alpha_{34} \cdot 1 + \alpha_{35} \cdot (-1) &= 1
    \end{align*}
    Opět stačí najít alespoň jedno řešení této soustavy, například:
    $$(0, 0, 0, 1, 0, 0)$$
    Třetí bázová forma tedy může být:
    $$\alpha_3(\vec{u}) = u_4$$

    To k čemu předchozí teorie směřovala je, že když vezmeme nějakou libovolnou formu,
    která působí na $\mathcal{V}$, tak tuto libovolnou formu lze vyjádřit jako lineární
    kombinaci těchto bázových forem.

    Vezměmě formu
    $$\varphi(\vec{u})) = u_1 + 2u_2 + 3u_3 + 4u_4 + 5u_5$$
    A ukážeme, že $\varphi$ lze na $\mathcal{V}^*$ vyjádřit jako linearní kombinaci
    $\alpha_1$, $\alpha_2$ a $\alpha_3$.

    \begin{align*}
        \varphi &= c_1 \cdot \alpha_1 + c_2 \cdot \alpha_2 + c_3 \cdot \alpha_3\\
        kde:
        c_1 &= \varphi(\vec{v_1}) = 13\\
        c_2 &= \varphi(\vec{v_2}) = 8\\
        c_3 &= \varphi(\vec{v_3}) = -1\\
        \\
        \varphi &= 13 \cdot \alpha_1 + 8 \cdot \alpha_2 - \alpha_3\\
    \end{align*}

    Nyní uvažujme $\vec{v} = (4, 4, 2, 3, 7) \in \mathcal{V}$ pomocí původního předpisu můžeme
    $\varphi(\vec{v})$ vyjádřit takto:
    $$\varphi(\vec{v}) 4 + 2\cdot4 + 3\cdot2 + 4\cdot3 + 5\cdot7 = 4 + 8 + 6 + 12+35 = 65$$
    A s použitím nového předpisu takto:
    $$\varphi(\vec{v}) = 13 \cdot 4 + 8 \cdot 2 - 3 = 65$$

    Pomocí duální báze jsme se dostali ke stejnému výsledku. Takto můžeme vyjádřit opravdu
    libovolnou formu na příslušném vektorovém prostoru.
\end{example}

S linearními formami se můžeme setkat například v analýze. Diferenciál v bodě můžeme uvažovat
jako základní příklad lineárních forem. I když se tento \uv{jazyk} forem v analýze příliš nezažil,
ale $dx$ je vlastně lineární forma, která přiřazuje vektoru jeho první souřadnici a
$dy$ je vlastně lineární forma, která přiřazuje vektoru jeho druhou souřadnici, obecně $d_{x_i}$
je lineární forma přiřazujicí vektoru jeho i-tou souřadnici. Při integrování tak vlastně integrujeme
linearní formy a ne funkce jako takové.



