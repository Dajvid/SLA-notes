\section{Osmá přednáška}
\subsection{Geometrie}

\begin{example}[Příčka mimoběžek]
    Mějme přímky $p$ a $q$ a bod $K$:
    \begin{align*}
        p&:\;\;\;A=[2, 1, 1],\,B = [3, 2, 6]\\
        q&:\;\;\;C=[2, -5, -1],\,D = [0, -4, 3]\\
        K &= [-3, 0, 3]
    \end{align*}
    Bodem $K$ veďte příčku $r$ k mimoběžkám $p,\,q$.

    Pro oveření že $p$ a $q$ jsou mimoběžky vytvoříme matici $P$ a určíme její hodnost.
    \begin{align*}
        P &= \begin{pmatrix}
            \vec{AB}\\
            \vec{CD}\\
            \vec{AC}
        \end{pmatrix} =
        \begin{pmatrix}
            1 & 1 & 5\\
            -2 & 1 & 4\\
            0 & -6 & -2
        \end{pmatrix} \eqop{}
        \begin{pmatrix}
            1 & 1 & 5\\
            0 & 3 & 14\\
            0 & -6 & -2
        \end{pmatrix} \eqop{}
        \begin{pmatrix}
            1 & 1 & 5\\
            0 & 3 & 14\\
            0 & 0 & 26
        \end{pmatrix} = R\\
        h(R) &= 3 = h(P)
    \end{align*}
    Matice má plnou hodnost $3$ a určitě se tedy jedná o mimoběžky.

    Máme tedy dvě přímky, které jsou mimoběžné a bod K, nyní hledáme
    přímku, která prochází bodem $K$ a zároveň má nějaké body $P$, $Q$,
    které tvoří průsečíky s přímkami $p$, $q$.

   Pomocí vektoru $\vec{AB}$ a bodu $A$ vyjádříme bod $P$ pomocí parametru $t$:
   $$P = [2 + t, 1 + t, 1 + 5t]$$
   Podobně pomocí vektoru $\vec{CB}$ a bodu C vyjádříme bod $Q$ pomocí parametru $s$:
   $$Q = [2 - 2s, -5 +s, -1 + 4s]$$

   Nyní vyjádříme vektory $\vec{KP}$ a $\vec{KQ}$ o kterých víme, že mají být lineárně závislé.
   \begin{align*}
       \vec{KP} &= [5 + t, 1 + t, -2 + 5t]\\
       \vec{KQ} &= [5 - 2s, -5 + s, -4 + 4s]
   \end{align*}
   A z toho, že vektory $\vec{KP}$ a $\vec{KL}$ mají být lineárně závislé můžeme odvodit:
   \begin{align*}
       \vec{KP} &= l \cdot \vec{KQ}\\
       (5 + t, 1 + t, -2 + 5t) &= l \cdot (5 -2s, -5 + s, -4 + 4s)
   \end{align*}
    Uvážíme li proměnné $t, l, ls$, potom dostaneme soustavu 3 lineárních rovnic o
    třech neznámých.
    \begin{align*}
    \begin{pmatrix}[ccc|c]
        1 & -5 & 2 & -5\\
        1 & 5 & -1 & -1\\
        5 & 4 & -4 & 2
    \end{pmatrix}
    \end{align*}

    Z této soustavy nám k vyřešení úlohy stačí vyjádřit neznámou $k$, nebo $s$,
    k čemuž se nám hodí cramerovo pravidlo:
    \begin{align*}
        t &= \frac{
            \begin{vmatrix}
                -5 & -5 & 2\\
                -1 & 5 & -1\\
                2 & 4 & -4\\
            \end{vmatrix}
        }{
            \begin{vmatrix}
                1 & -5 & 2\\
                1 & 5 & -1\\
                5 & 4 & -4\\
            \end{vmatrix}
        } = \frac{100 - 8 + 10 - (20 + 20 -20)}{-20 + 8 + 25 - (50 -4 + 20)} =
        \frac{78}{-53}
    \end{align*}
    A můžeme vyjádřit vektoř $\vec{KT}$:
    $$KP = (\frac{265 - 82}{53}, \frac{53 - 82}{53}, \frac{-106 - 410}{53}) =
    (\frac{183}{53}, \frac{-29}{53}, \frac{-516}{53})$$
    Nás zajímá pouze směr tohoto vektoru, můžeme si pomocí něj tedy určit směrový vektor,
    který bude vhodně vynásobený:
    $$\vec{u} = (183, -29, -516)$$
\end{example}

\subsubsection{Úlohy o vzdálenostech}
Vzdálenost dvou bodů můžeme vyjádřit jako
$$d(A, B) = ||\varphi (A,B)|| = \sqrt{\pi(\varphi(A, B)),\, \varphi(A, B)} =
\sqrt{(b_1 - a_1)^2 + \ldots + (b_n - a_n)^2}$$

\subsubsection{Vzdálenost bodu a přímky}
Máme nějaký bod $A$ a přímku $p$ a chceme zjistit jejich vzájemnou
vzdálenost $d(A, p)$.
Pro tento problém existuje obecný vzorec. Nám jde ale spíše
o to, jak takový vzorec odvodit, než ho jen použít.

Možnost řešení: vezmeme rovinu $\rho$ která bude kolmá k $p$ a zároveň bude procházet
bodem $A$, najdeme průsečík $K$ roviny $\rho$ a přímky $p$ (který bude jediný) a spočítáme
vzdálenost bodů $K$ a $A$.

\subsubsection{Vzdálenost bodu a roviny}
V případě vzdálenosti bodu a roviny můžeme použít obdobný přístup.
Máme nějaký bod $A$ a rovinu $\rho$, chceme spočítat $d(A, \rho)$.

K rovině najdeme kolmou přímku $p$,
která prochází bodem $A$. Najdeme průsečík $K$ přímky $p$ a roviny $\rho$,
a spočítáme vzdálenost bodů $K$ a $A$.

\subsubsection{Vzdálenost dvou rovnoběžných přímek}
Máme dvě rovnoběžky $p$ a $q$ a chceme vypočítat jejich vzájemnou
vzdálenost $d(p, q)$.

Na jedné z rovnoběžek si můžeme zvolit libovolný bod a postupovat
stejně jako při výpočtu vzdálenosti bodu a přímky.

\subsubsection{Vzdálenost prímky od roviny}
Máme přímku $p$ a rovnoběžnou rovinu $\rho$ a chceme spočítat jejich vzájemnou
vzdálenost $d(p, \rho)$.

Na přímce $p$ můžeme zvolíme libovolný bod a postupovat stejně
jako při výpočtu vzdálenosti bodu a roviny.

\subsubsection{Vzdálenost dvou rovnoběžných rovin}
Máme dvě rovnoběžné roviny $\rho$ a $\sigma$ a chceme spočítat
$d(\rho, \sigma)$.

Na jedné z rovin si můžeme zvolit libovolný bod a postupovat
stejně jako při výpočtu vzdálenosti roviny a bodu.

\subsubsection{Vzdálenost dvou mimoběžných přímek}
Máme dvě mimoběžné přímky $p$ a $q$ a chceme spočítat jejich
vzájemnou vzdálenost $d(p, q)$.

Vezmeme dvě navzájem rovnoběžné roviny a v nichž leží přímky $p$ a $q$.
Normálový vektor rovin můžeme najít jako cross product směrových vektorů
přímek $p$ a $q$. A následně můžeme postupovat jako v případě vzdálenosti dvou
rovin.

\subsection{Úlohy o úhlech}
Nejprve si připomeňme co je to odchylka vektorů.
Odchylka vektorů $\vec{u}$ a $\vec{v}$ se definuje jako
$\varphi \in \langle 0, \pi \rangle$, které splňuje:
$$cos\,\varphi = \frac{\vec{u} \cdot
\vec{v}}{||\vec{u}|| \cdot ||\vec{v}||}$$

\subsubsection{Vzájemný úhel dvou přímek}
Úhel dvou různoběžných přímek budeme brát jako
menší z úhlů, které mezi sebou svírají, a můžeme ho určit jako:
$$cos\,\varphi = \frac{|\vec{u} \cdot \vec{v}|}{||\vec{u}|| \cdot ||\vec{v}||}$$
kde $||$ v čitateli vyjadřuje absolutní hodnotu.

V případě že přímky budou mimoběžné, tak jejich úhel vypočítáme
úplně stejně a vždycky to bude pravý úhel.

\subsubsection{Vzájemný uhel přímky a roviny}
Máme rovinu $\rho$ a přímu $p$, které tuto rovinu protíná.
A chceme spočítat jejich vzájemný úhel.

Zkonstruujeme pomocnou rovinu $\sigma$, která je kolmá na rovinu
$\varphi$ ve které leží přímka $p$. K vytvoření této roviny stačí vzít bod
na přímce $p$, směrový vektor přímky $p$ a normálový vektor roviny $\rho$.
Roviny $\rho$ a $\sigma$ se poté protnou v přímce $q$
A nyní stačí spočítat vzájemný úhel přímek $p$ a $q$.

\subsubsection{Vzájemný úhel dvou rovin}
Úhle dvou roviny můžeme vypočítat jako odchylku
normálových vektorů, které vezmeme tak, aby jejich odchylka
byla $\leq \frac{\pi}{2}$

\subsubsection{Souhrnný příklad}
\begin{example}[Souhrnný příklad]
    Rovina $\rho$ prochází body $A = [2, 1, 2]$, $B = [4, 0, 1]$ a $C = [-3, 3, 4]$.
    Rovina $\sigma$ je rovnoběžná s $\rho$ a má od ní vzdálenost $4\sqrt{2}$.
    Na rovině $\sigma$ leží bod $S = [5, 0, ?]$

    Spočtěte vzdálenost $B$ a $S$.

    Nejprve si vyjádříme obecnou rovnici roviny $\rho$:
    \begin{align*}
        x &= 2 + 2t -5s \\
        y &= 1 - 1t + 2s\\
        z &= 2 - 1t + 2s\\
        \\
        y - z +1 &= 0
    \end{align*}
    Normálový vektor $\vec{n}$ k rovině $\sigma$ je tedy $\vec{n} = (0, 1, -1)$ a
    rovinu $\rho$ pomocí něj můžeme vyjádřit jako:
    $$y - z + d = 0$$
    Musíme však ještě vyjádřit neznámou $d$.

    Normálový vektor $\vec{n}$ upravíme tak, aby jeho délka byla rovna vzdálenosti obou rovin.
    \begin{align*}
        ||\vec{n}|| &= \sqrt{0^2 + 1^2 + (-1)^2} = \sqrt{2}\\
        \vec{n_1} &= 4 \cdot \vec{n} = (0, 4, -4)\\
        \vec{n_2} &= -4 \cdot \vec{n} = (0, -4, 4)\\
    \end{align*}
    Nyní přičteme $\vec{n_1}$ k nějakému bodu $E = [0, 0, 1]$, který leží na $\rho$ a dostaneme bod,
    který leží na $\sigma$:
    \begin{align*}
        S_1 &= R + \vec{n_1} = [0, 4, -3]\\
        S_2 &= R + \vec{n_2} = [0, -4, 5]
    \end{align*}
    A dosazením do rovnice pro rovinu $\sigma$ můžeme získat hodnotu neznámé $d$, pro
    který máme dvě možná řešení.
    \begin{align*}
       \sigma_1&:\; y - z - 7 = 0\\
       \sigma_2&:\; y - z + 9 = 0
    \end{align*}
    Nyní můžeme podle obecné rovnice pro rovinu $\sigma$ vyjádřit bod $S$.
    \begin{align*}
        S_1 &= [5, 0, -7]\\
        S_2 &= [5, 0, 9]
    \end{align*}

    A teď už jen spočítáme vzdálenosti bodů jak je požadování v zadání.
    \begin{align*}
        d(B, S_1) &= \sqrt{1 + 0 + 64} = \sqrt{65}\\
        d(B, S_1) &= \sqrt{1 + 0 + 64} = \sqrt{65}
    \end{align*}
\end{example}

\subsection{Formy}
