\section{Šestá přednáška}
\subsection{Matice}

\begin{theorem}[Cauchyho věta o součinu]
    \label{the:cauchy}
    Determinant součinu je součin determinantů.

    $$|A \cdot B| = |A| \cdot |B|$$
\end{theorem}
\begin{proof}
    Idea důkazu: Uvažujme matici H v tomto tvaru:
    \[H =
    \begin{pmatrix}
        A & O \\
        \begin{pmatrix}
            -1 & \ldots & 0\\
            \vdots & \ddots & \vdots \\
            0 & \ldots & -1
        \end{pmatrix} & B
    \end{pmatrix}
    \]
    Jestliže matice $A$ a $B$ jsou řádu $n$, potom matice $H$ je řádu $2\cdot n$.
    Nyní zkonstruujeme matici K:
    \[K =
        \begin{pmatrix}
            A & C \\
            \begin{pmatrix}
                -1 & \ldots & 0\\
                \vdots & \ddots & \vdots \\
                0 & \ldots & -1
            \end{pmatrix} & O
        \end{pmatrix}
    \]
    S tím, že:
    $$C = A\cdot B$$
    Zřejmě:
    \begin{align*}
    |H| &= |A| \cdot |B|\\
    |K| &= |C|
    \end{align*}
    A $K$ lze obdržet vhodnými elementárními úpravami z $H$.
\end{proof}

\subsection{Soustavy lineárních rovnic}
Příklady na řešitelnost soustavy. Příklad na soustavu, která nemá řešení je na konci
předchozí přednášky.

\begin{example}[Řešení soustavy linearních rovnic s právě jedním řešením]
    Řešte následujicí soustavu linearních rovnic:

    \begin{align*}
        2x + y - z & = 43\\
        3x + 3y +5z &= 84\\
        4x - y - 2z &= 86
    \end{align*}

    Soustavu přepíšeme do rozšířené matice a elementárními úpravami
    převedeme do schodovitého tvaru:
    \begin{align*}
        \begin{pmatrix}[ccc|c]
            2 & 1 & -1 & 43\\
            3 & 3 & 5 & 84\\
            4 & -1 & -2 & 86
        \end{pmatrix} \eqop{r_2  \cdot (-2); r_3 \cdot (-1)}
        \begin{pmatrix}[ccc|c]
            2 & 1 & -1 & 43\\
            -6 & -6 & -10 & -168\\
            -4 & 1 & 2 & 86
        \end{pmatrix} \eqop{r_2 + 3 \cdot r_1; r_3 + 2 \cdot r_1}
        \begin{pmatrix}[ccc|c]
            2 & 1 & -1 & 43\\
            0 & -3 & -13 & -39\\
            0 & 3 & 0 & 0
        \end{pmatrix} \eqop{r_2 + r_3; r_3 \cdot \frac{1}{3}; r_3 \leftrightarrow r_2; } \\
        \eqop{r_2 + r_3; r_3 \cdot \frac{-1}{3}; r_3 \leftrightarrow r_2; }
        \begin{pmatrix}[ccc|c]
            2 & 1 & -1 & 43\\
            0 & 1 & 0 & 0\\
            0 & 0 & -13 & -39
        \end{pmatrix} \eqop{r_3 \cdot \frac{-1}{13}}
        \begin{pmatrix}[ccc|c]
            2 & 1 & -1 & 43\\
            0 & 1 & 0 & 0\\
            0 & 0 & 1 & 3
        \end{pmatrix}
    \end{align*}
    Hodnost základní matice je 3, hodnost rozšířené matice je také 3, soustava má tedy právě jedno řešení.
    Nyní provedeme \uv{zpětný chod} a ze schodovitého tvaru matice vyjádříme odspodu hodnoty
    jednotlivých proměnných.
    \begin{align*}
        z &= 3\\
        y &= 0\\
        2x + y -z &= 43\\
        2x +0 -3 &= 43\\
        2x &=46\\
        x &= 23
    \end{align*}
\end{example}

\begin{example}[Řešení soustavy linearních rovnic s nekonečným počtem řešení]
    Řešte následujicí soustavu linearních rovnic v $\mathbb{R}$:

    \begin{align*}
        2a - b + c -4d &= -2\\
        4a + 2b -c + 5d &= 1\\
        10a -b +2c -7d &= -5\\
        30a + 9b -3c +18d &= 0
    \end{align*}
    Soustavu přepíšeme do rozšířené matice a elementárními úpravami
    převedeme do schodovitého tvaru:
    \begin{align*}
        \begin{pmatrix}[cccc|c]
            2 & -1 & 1 & -4 & -2\\
            4 & 2 & -1 & 5 & 1\\
            10 & -1 & 2 & -7 & -5\\
            30 & 9 & -3 & 18 & 0
        \end{pmatrix} \eqop{r_2 \cdot (-1); r_3 \cdot -1; r_4 \cdot \frac{-1}{3}}
        \begin{pmatrix}[cccc|c]
            2 & -1 & 1 & -4 & -2\\
            -4 & -2 & 1 & -5 & -1\\
            -10 & 1 & -2 & 7 & 5\\
            -10 & -3 & 1 & -6 & 0
        \end{pmatrix} \eqop{r_2 + 2 \cdot r_1; r_3 + 5 \cdot r_1; r_4 + 5 \cdot r_1}\\
        \eqop{r_2 + 2 \cdot r_1; r_3 + 5 \cdot r_1; r_4 + 5 \cdot r_1}
        \begin{pmatrix}[cccc|c]
            2 & -1 & 1 & -4 & -2\\
            0 & -4 & 3 & -13 & -5\\
            0 & -4 & 3 & -13 & -5\\
            0 & -8 & 6 & -26 & -10
        \end{pmatrix} \eqop{r_3 - r_2; r_4 - 2\cdot r_2}
        \begin{pmatrix}[cccc|c]
            2 & -1 & 1 & -4 & -2\\
            0 & -4 & 3 & -13 & -5\\
            0 & 0 & 0 & 0 & 0\\
            0 & 0 & 0 & 0 & 0
        \end{pmatrix}
    \end{align*}

    Vidíme, že hodnost základní matice soustavy je 2 a hodnost rozšířené matice soustavy je také 2
    hodnosti se rovnají a soustava je tedy řešitelná. Máme ovšem 4 neznáme a hodnosti jsou rovny
    dvěma, řešení lze tedy vyjádřit pomocí $4-2=2$ parametrů.

    Za neznámé $c, d$ dosadíme parametry $p, q$ a pomocí těchto parametrů opět zpětným
    chodem vyjádříme ostatní neznámé.
    \begin{align*}
        d &= q\\
        c &= p\\
        p, q &\in \mathbb{R}
    \end{align*}
    \begin{align*}
        -4 b +3 c -13d &= -5\\
        -4b +3p -13q &= -5\\
        -4b &= -3p + 13q -5\\
        b &= \frac{3p - 13q + 5}{4}
    \end{align*}
    \begin{align*}
        2a -b +c -4d &= -2\\
        2a -\frac{3p - 13q + 5}{4} + p -4q &= -2\\
        2a &= \frac{3p - 13q + 5}{4} + \frac{-4p+16q-8}{4}\\
        2a &= \frac{-p +3q -3}{4}\\
        a &= \frac{-p +3q -3}{8}
    \end{align*}
    Jako parametry ovšem nebylo nutné zvolit zrovna neznámé $c, d$. Jako parametr jsme mohli vzít
    například $a$, nebo $b$, nebo také výraz $a + 2b$. Můžeme si je zkrátka zvolit dle potřeby a
    parametrické vyjádření tedy není jednoznačné. Ne vždy je navíc možné parametry volit takto
    vhodně \uv{odzadu}, jak jsme to udělali v tomto příkladu.
\end{example}

\begin{example}[Případ, kdy nejde neznámé parametrizovat odzadu]
    Mějme rozšířenou matici soustavy upravenou na schodovitý tvar:
    \[
        \begin{pmatrix}[cccc|c]
            1 & 1 & 1 & 2 & 3\\
            0 & 0 & 0 & 2 & 5\\
            0 & 0 & 0 & 0 & 0\\
            0 & 0 & 0 & 0 & 0\\
        \end{pmatrix}
    \]
    V tomto případě si pro parametrizaci nemůžeme jednoduše zvolit poslední neznámou,
    protože ta má přesně danou hodnotu 5, k parametrizaci tak musíme využít jiný vhodný postup.
\end{example}

\begin{example}[Vytvoření soustavy rovnic dle požadavků]
    Vytvořte soustavu tří různých linearnách rovnic o dvou neznámých, která má
    nekonečně mnoho řešení.

    \begin{align*}
        x &= 0\\
        2y &= 0\\
        3y &= 0
    \end{align*}
\end{example}


\subsection{Výpočet inverzní matice pomocí Gaussovy metody}
%01:00:00 příklad na výpočet inverzní matice pomocí adjungované matice
Gaussovu metodu můžeme použít i k výpočtu inverzní matice. Pro výpočet
inverzní matice k matici $A$ stačí vytvořit rozšířenou matici:
\[
    \begin{pmatrix}[c|c]
    A & E
    \end{pmatrix}
\]
Kde matice E představuje jednotkovou matici. A pomocí elementárních úrav tuto rozšířeno
matici upravit do tvaru, kdy jednotkovou matici dostaneme na levé straně, inverzní
matice pak bude na pravé straně:
\[
    \begin{pmatrix}[c|c]
        E & A^{-1}
        \end{pmatrix}
\]
Tato úprava je realizovatelná v případě, že je matice $A$ regulární a existuje k ní tedy inverzní
matice.

\begin{example}[Výpočet inverzní matice Gaussovou metodou]
    Gaussovou metodou spočítejte inverzní matici $A^{-1}$ k matici $A$.
    \[A=
    \begin{pmatrix}
        2 & 3 & 0\\
        1 & 0 & -1\\
        1 & 1 & 5
    \end{pmatrix}
    \]

    Matici přepíšeme do rozšířené matice s jednotkovou maticí na pravé straně a levou
    stranu pomocí elementárních úprav převedeme na matici jednotkovou.
    \begin{align*}
        \begin{pmatrix}[ccc|ccc]
            2 & 3 & 0 & 1 & 0 & 0\\
            1 & 0 & -1 & 0 & 1 & 0\\
            1 & 1 & 5 & 0 & 0 & 1
        \end{pmatrix} \eqop{r_1 \leftrightarrow r_2; r_3 \cdot (-1)}
        \begin{pmatrix}[ccc|ccc]
            1 & 0 & -1 & 0 & 1 & 0\\
            2 & 3 & 0 & 1 & 0 & 0\\
            -1 & -1 & -5 & 0 & 0 & -1
        \end{pmatrix} \eqop{r_2 - 2\cdot r_1; r_3 + r_1}  \\ \eqop{r_2 - 2\cdot r_1; r_3 + r_1}
        \begin{pmatrix}[ccc|ccc]
            1 & 0 & -1 & 0 & 1 & 0\\
            0 & 3 & 2 & 1 & -2 & 0\\
            0 & -1 & -6 & 0 & 1 & -1
        \end{pmatrix} \eqop{r_2 + 3\cdot r_3; r_2 \leftrightarrow r_3}
        \begin{pmatrix}[ccc|ccc]
            1 & 0 & -1 & 0 & 1 & 0\\
            0 & -1 & -6 & 0 & 1 & -1\\
            0 & 0 & -16 & 1 & 1 & -3
        \end{pmatrix} \eqop{r_3 \cdot \frac{1}{16}} \\ \eqop{r_3 \cdot \frac{-1}{16}}
        \begin{pmatrix}[ccc|ccc]
            1 & 0 & -1 & 0 & 1 & 0\\
            0 & -1 & -6 & 0 & 1 & -1\\
            0 & 0 & 1 & \frac{-1}{16} & \frac{-1}{16} & \frac{3}{16}
        \end{pmatrix} \eqop{r_2 + 6 \cdot r_3; r_1 - r_3}
        \begin{pmatrix}[ccc|ccc]
            1 & 0 & 0 & \frac{-1}{16} & \frac{15}{16} & \frac{3}{16}\\
            0 & -1 & 0 & \frac{-6}{16} & \frac{10}{16} & \frac{2}{16}\\
            0 & 0 & 1 & \frac{-1}{16} & \frac{-1}{16} & \frac{3}{16}
        \end{pmatrix} \eqop{r_2 \cdot (-1)} \\ \eqop{r_2 \cdot (-1)}
        \begin{pmatrix}[ccc|ccc]
            1 & 0 & 0 & \frac{-1}{16} & \frac{15}{16} & \frac{3}{16}\\
            0 & 1 & 0 & \frac{6}{16} & \frac{-10}{16} & \frac{-2}{16}\\
            0 & 0 & 1 & \frac{-1}{16} & \frac{-1}{16} & \frac{3}{16}
        \end{pmatrix}
    \end{align*}
\end{example}

\subsection{Cramerovo pravidlo}
Cramerovo pravidlo je k řešení soustav lineárních rovnic, kde základní matice soustavy je
čtvercová regulární matice.

Cramerovo pravidlo říká, že i-tou neznámnou $x_i$ z matice soustavy $A$ můžeme vyjádřit jako:
$$x_i = \frac{|A_i|}{|A|}$$
Kde $A_i$ je matice vytvořená z matice $A$ tak, že i-tý sloupec nahradíme sloupcem absolutních
členů soustavy.

Tuto metodu je výhodné použít když nepotřebujeme všechny neznámé, nebo když máme \uv{nepěkné}
koeficienty.

\begin{example}[Výpočet neznámé pomocí cramerova pravidla]
    Pomocí Cramerova pravidla vyjádřete neznámou $y$ z následujicí soustavy rovnic:
    \begin{align*}
        \sqrt{2} x + y + z &= 2\sqrt{2}\\
        3x -\sqrt{2}z &= 1\\
        4y + 3\cdot \sqrt{2} z &= 1 + \sqrt{2}
    \end{align*}
    Spočítáme potřebné determinanty
    \begin{align*}
        |A| &=
        \begin{vmatrix}
            \sqrt{2} & 1 & 1\\
            3 & 0 & -\sqrt{2}\\
            0 & 4 & 3\cdot \sqrt{2}
        \end{vmatrix} =
        12 - (-4 \cdot \sqrt{2} \cdot \sqrt{2} + 9 \cdot \sqrt{2}) = 20 - 9 \cdot \sqrt{2} \\
        |A_2| &=
    \begin{vmatrix}
        \sqrt{2} & 2\cdot\sqrt{2} & 1\\
        3 & 1 & -\sqrt{2}\\
        0 & 1+\sqrt{2} & 3\cdot \sqrt{2}
    \end{vmatrix}
    = 9 + 3\cdot \sqrt{2} - [-2 - 2\sqrt{2} + 36] = -25 + 5 \sqrt{2}
      \end{align*}
    A pomocí nich vyjádříme neznámou $y$:
    $$y = \frac{|A_2|}{|A|} = \frac{-25 + 5 \cdot \sqrt{2}}{20 - 9 \cdot \sqrt{2}} \cdot
    \frac{20 + 9 \cdot \sqrt{2}}{20 + 9 \cdot \sqrt{2}} =
    \frac{-500 - 225\sqrt{2} + 100\sqrt{2} + 90}{238} = \frac{-410-125\sqrt{2}}{238}
    $$
\end{example}

\begin{definition}[Podobnost matic]
    %nástřel příkladů na podobnost matic 1:35
    Řekneme, že matice $B$ je podobná matici $A$, jestliže existuje regulární
    matice $S$ tak, že:
    $$B = S\cdot A \cdot S^{-1}$$
\end{definition}

\begin{theorem}[Relace podobnosti matic]
    Relace \uv{být podobná} je ekvivalence.
\end{theorem}
\begin{proof}
    Pro důkaz ekvivalence musíme dokázat, že se jedná o reflexivní, symetrickou a tranzitivní
    relaci:

    Reflexivita:
    $$A = S\cdot A\cdot S^{-1}\; pro\; S=E,  \forall A$$

    Symetrie: Předpokládejme, že B je podobná A:
    $$B = S\cdot A \cdot S^{-1}$$
    Potom musíme ukázat, že existuje regulární matice T taková, že:
    $$A = T \cdot B \cdot T^{-1}$$
    Pro splnění tohoto požadavku však stačí vzít $T = S^{-1}$

    Tranzitivita:
    Předpokládejme, že B je podobná A a že C je podobná B.
    \begin{align*}
        B &= SAS^{-1}\\
        C &= TBT^{-1}
    \end{align*}
    A nyní chceme ukázat, že C je podobná A.
    \begin{align*}
        C &= UAU^{-1}\\
        C &= TSAS^{-1}T^{-1}
    \end{align*}
    Pro splnění tohoto požadavku stačí zvolit $U = TS$
\end{proof}

\subsection{Charakteristický polynom, vlastní čísla a vlastní vektory matice}
% 1:40 opáčko soustav rovnic
\begin{definition}[Charakteristický polynom]
    Charakteristický polynom čtvercové matice $A$ je polynom:
    $$|A - \lambda E|$$
\end{definition}

\begin{definition}[Charakteristická rovnice]
    Charakteristická rovnice čtvercové matice $A$ je rovnice, kde charakteristický polynom
    položíme roven 0:
    $$|A - \lambda E| = 0$$
\end{definition}

\begin{definition}[Vlastní hodnoty]
    Vlastní hodnoty čtvercové matice $A$ jsou kořeny charakteristického polynomu této matice.
\end{definition}

\begin{example}
    Vyjádřete charakteristický polynom a vlastní hodnoty matice $A$.
    \[A=
    \begin{pmatrix}
        5 & 1\\
        -1 & 0
    \end{pmatrix}
    \]
    \begin{align*}
        |A - \lambda E| &= 0\\
        det \Bigg(
        \begin{pmatrix}
            5 & 1\\
            -1 & 0
        \end{pmatrix} - \lambda \cdot
        \begin{pmatrix}
            1 & 0\\
            0 & 1
        \end{pmatrix} \Bigg ) &= 0 \\
        det \Bigg(
        \begin{pmatrix}
            5 & 1\\
            -1 & 0
        \end{pmatrix} -
        \begin{pmatrix}
            \lambda & 0\\
            0 & \lambda
        \end{pmatrix} \Bigg ) &= 0 \\
        \begin{vmatrix}
            5 - \lambda & 1\\
            -1 & - \lambda
        \end{vmatrix} &= 0 \\
        -5 \cdot \lambda + \lambda^2 + 1 &= 0\\
        \lambda^2 -5 \cdot \lambda + 1 &= 0\\
        \lambda_{1,2} &= \frac{5 \pm \sqrt{21}}{2}
    \end{align*}
\end{example}

\begin{definition}[Vlastní vektor matice A]
    Vezmeme li jednu konkrétní vlastní hodnotu $\lambda_1$ matice $A$.
    Potom nenulový vektor $\vec{u}$, který splňuje:
    $$(A - \lambda_1 \cdot E) \cdot \vec{u} = \vec{o}$$
    Nazýváme vlastní vektor matice $A$ odpovídajicí vlastní hodnotě $\lambda_1$.
    Tento vlastní vektor tvoří vektorový podprostor.
\end{definition}

\begin{example}[Vlastní hodnoty a vlastní vektory]
    Najděte všechny vlastní hodnoty a vlastní vektory matice A.
    \[A =
        \begin{pmatrix}
            1 & 0 \\
            0 & 2
        \end{pmatrix}
    \]
Vyjádříme vlastní čísla:
    \begin{align*}
        \begin{vmatrix}
            1 - \lambda & 0 \\
            0 & 2 - \lambda
        \end{vmatrix} &= (1 - \lambda)(2 - \lambda)\\
        \lambda_1 &= 1\\
        \lambda_2 &= 2
    \end{align*}

K vlastním číslům vyjádříme vlastní vektory:
Pro $\lambda_1$
\begin{align*}
    \begin{pmatrix}
        0 & 0\\
        0 & 1
    \end{pmatrix} \cdot
    \begin{pmatrix}
        u_1\\
        u_2
    \end{pmatrix} &=
    \begin{pmatrix}
        0 \\
        0
    \end{pmatrix}\\
    u_1 &= p\\
    u_2 &= 0\\
\end{align*}
Pro $\lambda_2$ stejným postupem:
\begin{align*}
    u_1 &= 0\\
    u_2 &= p
\end{align*}
A vlastní vektory matice A jsou tedy $
\begin{pmatrix}
    p\\0
\end{pmatrix}$ a $
\begin{pmatrix}
    0\\p
\end{pmatrix}
$, kde $p\in \mathbb{R} \smallsetminus 0$
\end{example}

\begin{theorem}[Nezávislost vlastních vektorů]
    Vlastní vektory příslušné různým vlastním hodnotám jsou lineárně nezávislé.
\end{theorem}

\begin{theorem}[Vlastní hodnoty symetrické matice]
    Vlastní hodnoty symetrické matice $A$ jsou reálné ($\in \mathbb{R}$).
\end{theorem}

\begin{theorem}[Charakteristický polynom podobných matic]
    Charakteristický polynom podobných matic je stejný a tedy mají i stejné vlastní hodnoty.
    %důkaz 2:06
\end{theorem}

\subsection{Matice homomorfismu}
\begin{definition}[Matice homomorfismu]
    Mějme vektorový prostor $\mathcal{V}$ nad nějakým polem $F$, dimenze $\mathcal{V}$ bude
    $m$ a báze $\mathcal{B} = (\vec{e_1}, \ldots, \vec{e_m})$. A vektorový prostor $\mathcal{W}$
    s dimenzí $n$ a bází $\overline{B} = (\vec{f_1}, \ldots, \vec{f_n})$.

    Dále budeme mít zobrazení $\varphi: \mathcal{V} \rightarrow \mathcal{W}$, které bude homomorfismus
    (lineární zobrazení)
    Označme:
    \begin{align*}
        \vec{w_1} &= \varphi(\vec{e_1})\\
        &\vdots\\
        \vec{w_m} &= \varphi(\vec{e_m})\\
    \end{align*}
    Pak vektory $\vec{w_1}, \ldots, \vec{w_m}$ jsou lineárnímí kombinacemi bázových vektorů v prostoru
    $\mathcal{W}$.
    \begin{align*}
        \vec{w_1} &= a_{11} \cdot \vec{f_1} + \ldots a_{1n} \cdot \vec{f_n}\\
        &\vdots\\
        \vec{w_m} &= a_{n1} \cdot \vec{f_1} + \ldots a_{mn} \cdot \vec{f_n}\\
    \end{align*}
    Potom matici koeficientů $a$:
    \[
        \begin{vmatrix}
            a_{11} & \ldots & a_{1n} \\
            \vdots & \ddots & \vdots\\
            a_{n1} & \ldots & a_{mn}
        \end{vmatrix}
    \]
    Nazveme matice homomorfismu $\varphi: \mathcal{V} \rightarrow \mathcal{W}$ v bázích $\mathcal{B}$
    a $\overline{\mathcal{B}}$.
\end{definition}

\begin{example}[Matice homomorfismu]
    Nechť:
    \begin{align*}
        \mathcal{V} &= \mathbb{R}^3,\;\mathcal{B} = \big((1,1,1), (1,0,1), (1,0,0) \big)\\
        \mathcal{W} &= \mathbb{R}^2,\;\overline{\mathcal{B}} = \big((2,3), (3,2) \big)\\
        \varphi \big((\vec{v_1}, \vec{v_2}, \vec{v_3})\big) &= (0, \vec{v_2} + 2\cdot \vec{v_3})
    \end{align*}
    Potom:
    \begin{align*}
        \vec{w_1} &= (0, 3) = a_{11} \cdot (2, 3) + a_{12}\cdot (3,2)\\
        \vec{w_2} &= (0, 2) = a_{21} \cdot (2, 3) + a_{22}\cdot (3,2)\\
        \vec{w_3} &= (0, 0) = a_{31} \cdot (2, 3) + a_{32}\cdot (3,2)
    \end{align*}
    Vyřešíme tyto 3 soustavy lineárních rovnic a tím dostaneme následujicí matici homomorfismu:
    $$\begin{pmatrix}
       0 & 0\\
       \frac{6}{5} & \frac{-4}{5}\\
       0 & 0\\
    \end{pmatrix}$$
\end{example}