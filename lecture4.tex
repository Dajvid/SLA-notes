\section{Čtvrtá přednáška}
\subsection{Vektorové prostory}

\begin{definition}[Lineární obal]
    Mějme množinu $M \subseteq \mathcal{V}$

    Budeme uvažovat průnik všech vektorových podprostorů vektorového prostoru $\mathcal{V}$,
    které obsahují $M$. Množinu která z těchto průniků vznikne označíme jako lineární obal
    množiny $M$.

    Lineární obal množiny $M$ budeme označovat jako $\langle M \rangle$
\end{definition}

\begin{definition}[Lineární kombinace]
    Mějme nějaké vektory:
    $$\vec{u_1}, \ldots, \vec{u_n}$$
    Potom můžeme uvažovat jiný vektor $\vec{v}$ ve tvaru:
    $$\vec{v} = c_1 \cdot \vec{u_1} + \ldots + c_n \cdot \vec{u_n}$$
    Takovému vektoru $v$ potom říkáme lineární kombinace věktorů $u_1, \ldots, u_n$

    Zároveň platí, že vektor $\vec{v}$ je lineárně zásivlý na vektorech
        $\vec{u_1}, \ldots, \vec{u_n}$
\end{definition}

\begin{theorem}[Rovnost množiny všech lin. kombinací a lineárního obalu]
    Mějme množinu $M \subseteq \mathcal{V}$

    Pro zjednodušení označme množinu všech lineárních kombinací
    vektorů z množiny $M$ jako $lcM$\footnote{Pouze dočasně, toto
    označení nebudeme běžně používat.}

    Potom tvrdíme:
    $$\langle M \rangle = lcM$$
\end{theorem}
\begin{proof}
    Chceme ukázat, že platí:
    $$\langle M \rangle \subseteq lcM \wedge \langle M \rangle \supseteq lcM$$

    Pomocné tvrzení: $lcM$ je vektorový podprostor:
    \begin{enumerate}
        \item Sečtením dvou lineárních kombinací z $lcM$ dostaneme opět lineární kombinaci z $lcM$.
        \item Vynásobením lineární kombinace z $lcM$ skalárem dostáváme lineární kombinaci z $lcM$.
    \end{enumerate}
    $lcM$ je tedy vektorový podprostor. A pro každý vektor $\vec{v} \in M$ určitě existuje
    lineární kombinace $\vec{c}$ taková, že $\vec{v} = \vec{c} \cdot lcM$ tedy určitě
    obsahuje $M$.

    Důkaz pro $\langle M \rangle \subseteq lcM$: plyne z toho, že $lcM$ je vektorový
    podprostor obsahujicí $M$ a z toho, že $\langle M \rangle$ je průnik všech takových vektorových
    podprostorů. A průnik je určitě podmnožinou.

    Důkaz pro $\langle M \rangle \supseteq lcM$:
    $$\vec{v} \in lcM \Rightarrow \vec{v} = c_1
        \cdot \vec{u_1} + \ldots + c_n \cdot \vec{u_n}, \vec{u} \in M$$
    Potom:
    $$\vec{u_i} \in \mathcal{W}\; \forall\,\text{vektorové podprostory}\;
        \mathcal{W} \subseteq \mathcal{V}$$
    $$c_1 \cdot \vec{u_1} + \ldots + c_n \cdot \vec{u_n} \; \forall
    \, \text{vektorové podprostory}\; \mathcal{W} \subseteq \mathcal{V}$$
    to znamená, že $\vec{v} \in \bigcap\limits_{i} \mathcal{W}_i$
\end{proof}

Označení $lcM$ nebudeme nadále používat, protože jak jsme ukázali, jedná se vlastně
o ekvivalentní definici lienárního obalu.

\begin{definition}[Lineární nezávislost vektorů]
    \label{def:lin_nezavislost}
    Uvažujme množinu vektorů:
    $$(\vec{u_1}, \ldots, \vec{u_n})$$
    Řekneme, že vektory $\vec{u_1}, \ldots, \vec{u_n}$ jsou lineárně nezávislé,
    jestliže:
    $$c_1\cdot\vec{u_1}+\ldots +c_n\cdot\vec{u_n} = \vec{o} \Rightarrow \forall c_i = 0$$
\end{definition}

\begin{definition}[Báze vektorového podprostoru]
    Báze $\mathcal{B}$ vektorového podprostoru $\mathcal{W}$ je uspořádaná n-tice lineárně
    nezávislých vektorů, které generují $\mathcal{W}$.
    Kde \textit{generují} znamená, že $\langle \mathcal{B} \rangle = \mathcal{W}$.
\end{definition}

\begin{example}[Příklad báze vektorového podprostoru]
    $$\mathcal{W} = \mathcal{V} = \mathbb{R}^3$$
    $$\mathcal{B} = \big((1, 0, 0), (0, 1, 0), (0, 0, 1)\big)$$
    Je $\mathcal{B}$ báze vektorového podprostoru $\mathcal{W}$?
    \begin{enumerate}
        \item $(a, b, c) = a \cdot(1, 0, 0) + b \cdot (0, 1, 0) + c \cdot (0, 0, 1)$ \hfill
            $\mathcal{B}$ generuje celé $\mathcal{W}$
        \item $c_1 \cdot(1, 0, 0) + c_2 \cdot (0, 1, 0) + c_3 \cdot (0, 0, 1) = (0, 0, 0)
            \Rightarrow c_1 = 0, c_2 = 0, c_3 = 0$ \hfill $\mathcal{B}$ je lineárně nezávislé
    \end{enumerate}
    A $\mathcal{B}$ je tedy báze vektorového podprostoru $\mathcal{W}$.
\end{example}

\begin{definition}[Dimenze vektorového prostoru]
    Počet prvků báze $\mathcal{B}$ budeme označovat jako dimenzi vektorového
        prostoru $\langle \mathcal{B} \rangle$
\end{definition}

\begin{example}[Báze vektorového prostoru matic] Uvažujme:
    $$\mathcal{V} = Mat_{2,3}(\mathbb{R})$$
    Potom bázi můžeme určit jako:
    \[
        \mathcal{B} = \Bigg(
            \begin{pmatrix}
                1 & 0 & 0 \\
                0 & 0 & 0
            \end{pmatrix},
            \begin{pmatrix}
                0 & 1 & 0 \\
                0 & 0 & 0
            \end{pmatrix},
            \begin{pmatrix}
                0 & 0 & 1 \\
                0 & 0 & 0
            \end{pmatrix},
            \begin{pmatrix}
                0 & 0 & 0 \\
                1 & 0 & 0
            \end{pmatrix},
            \begin{pmatrix}
                0 & 0 & 0 \\
                0 & 1 & 0
            \end{pmatrix},
            \begin{pmatrix}
                0 & 0 & 0 \\
                0 & 0 & 1
            \end{pmatrix}
            \Bigg)
    \]
    A vidíme, že dimenze tohoto vektorového prostoru je 6.

    Což dává smysl mimo jiné díky tomu, že v případě $Mat_{2,3}$ v podstatě pracujeme s
    $\mathbb{R}^6$ pouze s tím, že jsme prvky uspořádali do obdélníku. Tato změna uspořádání
    nemá z hlediska aditivní grupy a vektorového prostoru jako takového žádný zvláštní
    význam a změna se projeví až ve chvíli, kdy začneme matice násobit.
\end{example}

% 0:56 příklad na bázi C<0,1>

\subsubsection{Speciální zobrazení mezi vektorovými prostory}

\begin{definition}[Homomorfismus vektorovych prostorů]
    Uvažujme dva vektorové prostory $\mathcal{V}, \mathcal{W}$ a zobrazení $\varphi$:
    $$\varphi = \mathcal{V} \rightarrow \mathcal{W}$$
    kde $\varphi$ bude mít následujicí vlastnosti:
    \begin{enumerate}
        \item $\varphi(\vec{v_1} + \vec{v_2}) = \varphi(\vec{v_1}) + \varphi(\vec{v_2})\;
            \forall \vec{v_1}, \vec{v_2} \in \mathcal{V}$ \hfill Zachování součtu
        \item $\varphi(k\cdot v) = k \cdot \varphi (\vec{v}) \;
            \forall \vec{v} \in \mathcal{V} \; \forall k \in F$ \hfill Zachování násobení skalárem
    \end{enumerate}
    Neformálně řeceno $\varphi$ je zobrazení, které zachovává operace.

    Potom zobrazení $\varphi$ nazýváme homomorfismus\footnote{Někdy se také používá pojem
    lineární zobrazení.} (vektorových prostorů).
    \label{def:homo}
\end{definition}

\begin{example}[Homomorfismus vektorových prostorů]
    $$\mathcal{V} = \mathbb{R}^2, \mathcal{W} = \mathbb{R}^4$$
    $$\varphi(\vec{v}) = \vec{w}$$
    $$\vec{w} = (v_1, 0, v_1, v_1 + v_2)$$
    $$\varphi\big((1,2)\big) = (1, 0, 1, 3)$$

    Je takto definované $\varphi$ homomorfismus? Musí platit podmínky z definice
    homomorfismu \ref{def:homo}.

    První podmínka:
    $$LS = \varphi(\vec{u} + \vec{v}) = \varphi(u_1 + v_1, 0, u_1 + v_1 + u_2 + v_2),
    \vec{u}, \vec{v} \in \mathbb{R}^2$$
    $$PS = \varphi(\vec{u}) + \varphi(\vec{v}) = (u_1, 0, u_1 + u_2) + (v_1, 0, v_1 + v_2) =
    (u_1 + v_1, 0, u_1 + u_2 + v_1 + v_2), \vec{u}, \vec{v} \in \mathbb{R}^2$$
    $$LS = PS$$

    První podmínka tedy platí a stejným postupem by bylo možné ukázat i platnost
    druhé podmínky, jedná se tedy o homomorfismus.
\end{example}

\begin{definition}[Jádro a obraz homomorfismu] Uvažujme vektorový prostor $\mathcal{V}$
    a homomorfismus $\varphi$, potom je kernel $\ker$ homomorfismu $\varphi$ definován takto:
    $$ker\,\varphi = \{\vec{v} \in \mathcal{V}; \varphi(\vec{v}) = \vec{o}\}$$
    A obraz $im$ homomorfismu $\varphi$ je definován takto:
    $$im\,\varphi = \{\vec{w} \in \mathcal{W}; \exists \vec{v} \in \mathcal{V}\,
        \text{tak, že}\, \varphi(\vec{v}) = \vec{w}\}$$
\end{definition}

\begin{theorem}[$ker\, \varphi$ a $im\, \varphi$ jsou vektorové podprostory $\mathcal{V}$ a $\mathcal{W}$
    v tomto pořadí]

    \begin{enumerate}
        \item[]
        \item $ker\,\varphi$ je vektorový podprostor $\mathcal{V}$,
            kde $\mathcal{V}$ je z definice \ref{def:homo}.
        \item $im\,\varphi$ je vektorový podprostor $\mathcal{W}$,
            kde $\mathcal{W}$ je z definice \ref{def:homo}.
    \end{enumerate}
\end{theorem}
\begin{proof}
    \begin{enumerate}
        \item[]
        \item $\vec{u}, \vec{v} \in ker\,\varphi:
            \varphi(\vec{u}) = \vec{o},\;\varphi(\vec{v}) = \vec{o}$ \\
            $\varphi(\vec{u} + \vec{v}) = \varphi(\vec{u}) + \varphi
            (\vec{u})\footnote{Vychází z vlastností homomorfismu.} =
            \vec{o} + \vec{o} = \vec{o}$
        \item $\vec{u} \in ker\,\varphi$ \\
              $\varphi(k \cdot \vec{u}) = k \cdot \varphi(u) \footnote{Z definice homomorfismu}
              = k \cdot \vec{o} = \vec{o}$
    \end{enumerate}
    Ukázali jsme, že $ker\,\varphi$ splňuje všechny podmínky k tomu, aby byl vektorový
    podprostor $\mathcal{V}$.

    Podobným způsobem bychom ukázali i druhou část věty o $im\,\varphi$ a došli také ke kladnému
    závěru.
\end{proof}

\begin{definition}[Monomorfismus]
    Jestliže je homomorfismus injektivní, nazýváme ho monomorfismus.
\end{definition}

\begin{definition}[Epimorfismus]
    Jestliže je homomorfismus surjektivní, nazýváme ho epimorfismus.
\end{definition}

\begin{definition}[Izomorfismus]
    Jestliže je homomorfismus bijektivní, nazýváme ho izomorfismus.
\end{definition}

\begin{definition}[Endomorfismus]
    Jestliže má homomorfismus $\varphi$ výchozí i cílovou množinu totožnou, tedy:
    $$\varphi: V \rightarrow V$$
    nazveme ho endomorfismus.

    Intuitivně můžeme říct, že se jedná o homomorfismus
    do sebe sama.
\end{definition}

\begin{definition}[Automorfismus]
    Homomorfismu, který je endomorfismem a současně izomofismem nazveme automorfismus.
\end{definition}

\begin{theorem}
    $$\varphi: \mathcal{V} \rightarrow \mathcal{W}\,\text{je epimorfismus}\,\Leftrightarrow
    im\,\varphi = \mathcal{W}
    $$
\end{theorem}

\begin{theorem}
    $$\varphi: \mathcal{V} \rightarrow \mathcal{W}\,\text{je monomorfismem}\,\Leftrightarrow
    ker\,\varphi = \{\vec{o}\}
    $$
\end{theorem}
\begin{proof}
    Dokažme, že $\varphi: \mathcal{V} \rightarrow \mathcal{W}$ není monomorfismem
    $\Leftrightarrow ker\,\varphi \neq \{\vec{o}\}$

    Důkaz pro $\varphi: \mathcal{V} \rightarrow \mathcal{W}$ není monomorfismem
    $\Rightarrow ker\,\varphi \neq \{\vec{o}\}$:

    Předpokládejme, že $\varphi$ nená monomorfismus, to znamená, že $\varphi$ není inejktivní.

    To, že $\varphi$ není injektivní znamená, že existují nějaké vektory $\vec{u}, \vec{v} \in
    \mathcal{V}$, pro které:
    $$\vec{u} \neq \vec{v}, \varphi(\vec{u}) = \varphi(\vec{v})$$
    potom
    $$\varphi(\vec{u} - \vec{v}) =\footnote{Tato rovnost vychází z definice homomorfismu.} \varphi(\vec{u}) - \varphi(\vec{u}) = \vec{o}$$
    Ovšem $\vec{u}$ je různé od $\vec{v}$ a tedy:
    $$\vec{u} - \vec{v} \in ker\,\varphi$$
    A $\vec{u} - \vec{v}$ je nulový vektor, takže jádro je netriviální.

    Důkaz pro $\varphi: \mathcal{V} \rightarrow \mathcal{W}$ není monomorfismem
    $\Leftarrow ker\,\varphi \neq \{\vec{o}\}$:

    Předpokládejme, že $im\, \varphi$ je netriviální a ukážeme, že pak zobrazení nemůže být
    homomorfismem.

    Z předpokladu netriviálního jádra:
    $$\exists \vec{v} \neq \vec{o}, \vec{v} \in \mathcal{V}\,
        \text{tak, že}\,\varphi(\vec{v}) = \vec{o}$$
    $$\vec{u} \in \mathcal{V}\; \varphi(\vec{u}) = \vec{w} \in \mathcal{W}$$
    $$\varphi(\vec{u} + \vec{v}) = \varphi(\vec{u}) + \varphi(\vec{v}) = \vec{w} + \vec{o} = \vec{w}$$
\end{proof}

\subsection{Matice}

\begin{definition}[Stopa]
    Stopa je definována pro čtvercové matice. Jedná se o zobrazení,
    které čtvercové matici přiřadí jedno číslo. Stopu budeme značit
    $tr$\footnote{Z anglického trace.}

    $$tr: Mat_n(F) \rightarrow F$$
    $$tr(A) = \sum_{i=1}^na_{ii}$$

    Vlastnosti:
    \begin{itemize}
        \item $tr(A^T) = tr(A)$
        \item $tr(A + B) = tr(A) + tr(B)$
        \item $tr(k \cdot A) = k \cdot tr(A)$
        \item $tr(A\cdot B) = tr(B \cdot A)$\footnote{Zajímavé však je, že
        $tr(ABC) \neq tr(ACB)$}
    \end{itemize}

\end{definition}
\begin{proof}
    ($tr(A\cdot B) = tr(B \cdot A)$)

    $$C = AB, D = BA$$
    $$tr(AB) = \sum_{i=1}^n\sum_{k=1}^n a_{ik} \cdot b_{ki}$$
    $$tr(BA) = \sum_{i=1}^n\sum_{k=1}^n b_{ik} \cdot a_{ki} = \sum_{i=1}^n\sum_{k=1}^n a_{ki}
    \cdot b_{ik} = \sum_{k=1}^n\sum_{i=1}^n a_{ik} \cdot b_{ki} =
    \sum_{i=1}^n\sum_{k=1}^n a_{ik} \cdot b_{ki}$$

\end{proof}

\begin{definition}[Determinant]
    \label{def:determinant}
    Determinant budeme definovat pro čtvercové matice.
    Opět se jedná o zobrazení, které čtvercové matici přiřadí
    jedno číslo. Determinant matice $A$ budeme značit $det\,A$, nebo také $|A|$.

    $$det: Mat_n(F) \rightarrow F$$
    $$det\, A = \sum_{\sigma \in S_n} sgn(\sigma) \cdot a_{1\sigma(1)} \cdot a_{2\sigma(2)}
    \cdot \ldots \cdot a_{n\sigma(n)}$$

    Pří výpočtu podle vzorce je vhodné postupovat tak, že u řádkových indexů, začneme
    od jedničky a postupně zvyšujeme řádkový index a u sloupcových indexů uvažujeme všechny možné
    permutace n prvkové množiny těchto sloupcových indexů, tím nám vznikne nějaká permutace, která
    je buďto sudá, nebo lichá a danému součinu přiřadíme znaménko permutace.
\end{definition}

\begin{example}[Výpočet determinantu pro matice řádu 1]
    $$n = 1: S_1=\{1\}$$
    $$det\,A = a_{11}$$
\end{example}

\begin{example}[Výpočet determinantu pro matice řádu 2]
    $$n = 2: S_2=\{(1,2), (2,1)\}$$
    $$det\,A = a_{11} \cdot a_{22} - a_{12} \cdot a_{21}$$
\end{example}

\begin{example}[Výpočet determinantu pro matice řádu 3]
    $$n = 3: S_3=\{(1,2,3), (1,3,2), (2,1,3), (2,3,1), (3,1,2), (3,2,1)\}$$

    $$det\,A = a_{11}\cdot a_{22}\cdot a_{33} +
               a_{12}\cdot a_{23}\cdot a_{31} +
               a_{13}\cdot a_{21}\cdot a_{32} -
               a_{11}\cdot a_{23}\cdot a_{32} -
               a_{12}\cdot a_{21}\cdot a_{33} -
               a_{13}\cdot a_{22}\cdot a_{31}
    $$
\end{example}

\begin{definition}[Schodovitá matice]
    Je horní trojúhelníková matice, ve které můžeme nulovou část oddělit "schody", které
    mají výšku stupně 1.
\end{definition}

